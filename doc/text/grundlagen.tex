\chapter{Grundlagen}

\section{Datenquellen}
Die Datengrundlage für dieses Projekt sind Daten des Mars Orbiter Laser Altimeter (MOLA), einem Höhenmessgerät an Bord des 
               

% Beschreibe die MOLA Mission und mit welchen Instrumenten sie erfasst wurden

% Beschreibe die Parameter (Auflösung, Datengröße, Genauigkeit (horizontal/vertikal))
% Beschreibe das Datenformat und Unterschiede / Vor- und Nachteile

%% https://pds-geosciences.wustl.edu/missions/mgs/megdr.html vs. https://astrogeology.usgs.gov/search/details/Mars/GlobalSurveyor/MOLA/Mars_MGS_MOLA_DEM_mosaic_global_463m
%% entscheide dich für die 2., auch wenn du das Format des 1. besser findest <-- da beim 1. Daten fehlen und das Mapping Auf Kugel dann schwierig wird
%% beschreibe, warum das Format nicht gut ist -> beschreibe Aufbau von TIFF Dateien (Container-Format, Daten an verschiedenen Stellen in der Datei) -> decoder benötigt

\section{OpenGL}
% Grundlagen: nur Interface, verschiedene Implementationen in Form von Driver
% existiert auch WebGL ← Unterschiede und Gemeinsamkeiten, z.B. 
% Erklärung Shader Pipeline
% Erklärung Kommunikation mit Shader (Attribut vs Uniform)
% Koodinaten Systeme -> wie verlaufen die Achsen (positive x-Achse nach rechts, positive y-Achse nach oben und positive z-Achse in den Bildschirm hinein)

\section{Datenreduzierung}

\subsection{Datenmenge}\label{datenmenge}

\subsection{Verlustfrei}

\subsection{Verlustbehaftet}
% Abschnitt, der erstmal die vollständige Datengröße beschreibt
% außerdem ist es ein Rasterformat, also feste Abstände zwischen einzelnen Datenpunkten

% verlustfrei: alles was nicht im Sichtbereich der Kamera liegt (Frustum Culling) und alles, was durch andere Dinge verdeckt wird (Occlusion Culling) -> andere Seite des Globus
% dazu Unterteilung der Welt in Abschnitte (chunks) und nur Anzeigen der Abschnitte, welche den Sichtbreich auch nur ansatzweise schneiden und nicht vollständig durch andere Abschnitte verdeckt sind
% verlustbehaftet: 
% hier ausnutzen, dass zum Beispiel bei geringen Zoomstufen Unterschiede in den Daten nicht mehr wahrgenommen werden können
% Außerdem: Entfernung von Redundanzen -> da Rasterformat , hier ist es abhängig von der Komplexität des Terrains,

% Redukttion der Datenmenge in festgelegten Schritten (stride), systemetaische Reduktion
% Reduktion der Datenmenge abhängig vom umliegenden Terrain -> ähnliche vertices müssen nicht wiederholt werden,
% gerade bei natürlichen Daten gibt es sehr ähnliche Pubkt

% eine Plane besteht zwar auf 4 Eckpunkten, allerdings muss sie in OpenGL Primitives zerlegt werden -> dies sind normalerweise Dreiecke, also 6 Eckpunkte, da ja jeder Eckpunk auch Teil des benachbartenn PLane ist, wird insgesamt ein Eckpunkt 6 mal wiederholt (siehe kleine Grafik) -> also wird bei der Erzeugung des Modell eine Indexierung vorgenommen, einfach eine Liste mit Indexen die eine die Liste mit den Vertex-Daten referenziert ->
% bei der systematischen Reduktion ist das kein Problem, die Indixierung des Raster Modells bleibt erhalten, nur die Abstände zwischen den Vertices erhöhen sich

% erwähne die Formel um aus einem 1D Array ein 2D Array zu erstellen

% bei der asymetrischen Reduktion kann nicht mehr aus dem 1D Vertex-Format auf eine Indexierung geschlossen werden, also muss hier ein 2D Format gesendet werden was die entfernen
% Vertex-Daten kennzeichnet