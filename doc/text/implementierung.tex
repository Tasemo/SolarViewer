\chapter{Implementierung}

\section{Build System}
Der Build Prozess ist ein Prozess, welcher alle Schritte beinhaltet um aus dem vorhandenen Source Code eine ausführbare Software zu erstellen. Insbesondere werden dabei benötigte Abhängigkeiten zur Verfügung gestellt und in die Software integriert, der Source Code kompiliert und alle Software Teile in einem ausführbaren Format zusammengeführt. Dieser Prozess kann sehr komplex werden und ist daher häufig fehleranfällig. Unter anderem die Kombination von mehreren Build Systemen und unterschiedlichen Programmierumgebungen wird dabei als ein Kernproblem angesehen\ref{buildSystemProblem}. Da sich dieses Projekt in ein Frontend und Backend mit unterschiedlichen Programmiersprachen gliedert, ist es allerdings notwendig unterschiedliche Build Systeme zu integrieren. Dies ist erforderlich, da die existierenden Systeme sich eher auf eine Umgebung spezialisieren und zum Beispiel nur Abhängigkeiten einer Programmiersprache in einem zentralen Repository anbieten (z.B. npm registry für JavaScript oder maven central für Java Anwendungen). Auch existieren bestimmte Plugins, zum Beispiel für das Transpilieren von TypeScript, nicht für alle Build Systeme. Alles in allem erhöht sich zwar die Build Komplexität und Dauer, allerdings hat es auch Vorteile. Zum einen führt es dazu, dass das Frontend vom Backend vollständig isoliert ist und die reine Frontendentwicklung ohne unnötige Abhängigkeiten und Build Prozesse ablaufen kann. Zum anderen erweitert sich dadurch auch die Auswahl an vorhanden Tools und Plugins, was die Entwicklung deutlich vereinfachen kann.

Für dieses Projekt wird Maven als System für das Backend und npm als System für das Frontend genutzt. Die Entscheidung wurde auf Grund der hohen Beliebtheit und der persönlichen Expertise mit diesen Tools getroffen. Aus reiner Systemsicht ist das Frontend ein Teil des Backends, da es für die Auslieferung des Frontends an den Nutzer verantwortlich ist. Der Build beginnt daher auch in Maven. Als erstes muss sichergestellt werden, dass npm und Node.js auf dem Rechner vorhanden sind. Für die Kommunikation mit diesen wird das frontend-maven-plugin\footnote{\url{https://github.com/eirslett/frontend-maven-plugin}} genutzt, welches sie in einem ersten Schritt in einen lokalen Ordner installiert, solange sie noch nicht vorhanden sind. Anschließend wird “npm install” aufgerufen, um alle deklarierten Abhängigkeiten aus dem Frontend zu installieren. Dazu zählt zum Beispiel Three.js oder auch der TypeScript Transpiler. Anschließend wird der Build an das Frontend übergeben, indem ein fest definiertes npm Script aufgerufen wird. Dieses führt als erstes den TypeScript Transpiler aus um die benötigten JavaScript Dateien anzulegen. Da npm eigentlich ein Build-Tool für Node.js, welches lokal in einem Server ausgeführt wird, müssen die Abhängigkeiten für die Verwendung im Browser besonders behandelt werden. Hier ist ein sogenannter Bundler erforderlich, welche alle Abhängigkeiten in einer lokalen Datei zusammenfasst, welche dann an den User ausgeliefert wird. Als erstes wurde das Tool Browserify in Betracht gezogen, was ein existierendes NodeJS-Projekt ohne Codeänderungen für den Browser vorbereitet. Es wurde sich jedoch für das Tool Webpack entschieden, hauptsächlich auf Grund des direkten Supports von JavaScript Modulen ohne weitere Abhängigkeiten. Die Nutzung eines Bundlers hat des Weiteren des Vorteil, dass der Code minifiziert wird und in einer einzigen Datei zusammengefasst wird, was ein deutlich schnelleres Laden der Webseite zur Folge hat. Im Gegenzug verliert man die Fähigkeit, den Frontend Code zu debuggen, da die Minifizierung dabei natürlich stark stört. Dies kann allerdings durch die Verwendung von Source Maps verhindert werden, die für genau diesen Zweck eine Zuweisung von minifizierten zu originalen Variablen- und Methodennamen zur Verfügung stellen. Um das Frontend jetzt per Webserver an den User auszuliefern, müssen die Dateien jetzt in einen Ordner im Java Classpath befördert werden. Wichtig dabei ist, dass nur die benötigten Dateien verschoben werden dürfen, da diese sonst über den Webserver erreichbar wären. Dies wäre insbesondere bei den diversen Konfigurationsdateien ein Sicherheitsproblem. Auch das Verschieben des node\_modules Ordner wäre ein grober Fehler, da dies zu einer relativ großen ausführbaren Datei führen würde. Anschließend geht der Build im Backend mit dem Kompilieren der Java Dateien weiter. Nachdem alle Tests automatisch durchlaufen wurden, werden dann alle Abhängigkeiten des Backends, unter anderem der Undertow Server, zu einem fertigen Artefakt (Fat JAR) zusammengeführt, welches dann auf einen Server deployed und dort ausgeführt werden kann.

\section{Visualisierung}
Für die eigentliche Visualisierung kommen Shader zum Einsatz. Dafür muss dem Shader natürlich der Höhenwert bekannt sein. Bei einer flachen Projektion lässt sich dieser aus dem Vertex ablesen (entsprechend skalierte y-Koordinate), dies ändert sich natürlich, sobald eine sphärische Projektion zum Einsatz kommt. Die einfachste Möglichkeit ist es, dem Shader den Höhenwert in Metern neben dem Vertex als weiteres Attribut zu übergeben. Da dies die theoretische maximale Speichernutzung allerdings um weitere 4 GB erhöhen würde, was einer Erhöhung um 1/3 der Gesamtmenge entspricht, wurde dieser Ansatz verworfen. Stattdessen ist effizienter, den Höhenwert aus dem Vertex zu berechnen. Da der Höhenwert immer der Abweichung von einem vordefinierten Radius entspricht (ähnlich dem Meeresspiegel auf der Erde), muss dieser Radius einfach von der Länge des Vertex abgezogen werden. Dies funktioniert allerdings nur, wenn der Mittelpunkt des Modells auch mit dem Koordinaten-Nullpunkt übereinstimmt, da nur dann die Länge des Vertex der Distanz zum Punkt auf der Oberfläche entspricht. Um dies im Vertexshader zu implementieren, müssen ihm die verwendete Projektion, der Radius und die Skalierung von GL Einheiten zu Metern bekannt sein. Diese werden als Uniform Werte übergeben, sind also nicht abhängig von der Vertex-Anzahl und spielen daher für die maximale Speichernutzung keine Rolle. Der Vertexshader berechnet also wie beschrieben die Höhe in Metern und gibt sie an den Fragmentshader weiter. Des Weiteren transformiert er wie üblich den Vertex mit der Model-Matrix (enthält die Transformationen des Modells), der View-Matrix (enthält die inversen Transformationen der Kamera) und der Projektions-Matrix (enthält die perspektivische Transformation der Kamera) um die endgültige Position des Vertex zu bestimmen.

Der Fragmentshader hat nun die Aufgabe, aus diesem Höhenwert einen Farbwert zu generieren. Eine Möglichkeit wäre es, einfach verschiedene Grenzen zu definieren und diesen Grenzen feste Farbwerte zuzuweisen. Dann kann geprüft werden, welchem Bereich der Höhenwert entspricht und der endgültige Farbwert entspricht dann einer Variation des Farbwerts des Bereichs. Da dies allerdings mehrere Verzweigungen (conditionals) zur Prüfung der Grenzen erfordert und dies in der Shaderentwicklung vermieden werden sollte\footnote{siehe \cite{shaderDev}, Kapitel 14, Abschnitt Avoid Dynamic Branching, S. 273}, wurde eine bessere Lösung gesucht. Insbesondere, da es sich um das sogenannte dynamic branching handelt, da da die Bedingung abhängig vom Höhenwert ist, welcher natürlich pro Vertex anders ist. Des Weiteren wurde auf Grund der Datenmenge die kritischste Stelle der Performance (bottleneck) eher auf GPU Seite angesehen, sodass hier dringender auf der Performance geachtet werden sollte. Schlussendlich wurde der Fakt genutzt, das der Hue-Wert im HSV-Farbraum eine relativ lineare Verteilung verschiedener Farben enthält und so gut als Farbskala genutzt werden kann. Da die Ausgabe des Shaders allerdings im RGBA-Format erfolgen muss, ist hier eine Umwandlung des HSV Werten in diesen Farbraum erforderlich. Es wird also der Prozentwert des aktuellen Höhenwerts abhängig von vom Nutzer definierten Grenzen berechnet und diesem Prozentwert ein Hue Wert zugeordnet. Dabei wurde der Farbraum vorher noch verkleinert und invertiert, sodass die Farben dann von einem Blau-Ton (niedrigster Wert) zu einem Rot-Ton (höchster Wert) reichen. 

\section{Kamerabewegung}
Die erste Implementierung war eine frei im Raum bewegbare Kamera, welche man mit der Tastatur steuern konnte. Als konkrete Tasten wurden zum einen die Pfeiltasten als auch die übliche Alternative WASD genutzt. Diese sind allgemein als Steuerungstasten bekannt und sollten daher keiner Erklärung bedürfen. Die Kamera bewegte sich dabei entlang des lokalen Koordinatensystems der Kamera. Dieses kann dann mit der Maus entlang der x-Achse (pitch) und y-Achse (yaw) gedreht werden. Eine Drehung um die z-Achse (roll) verkompliziert die Steuerung und wurde daher bewusst nicht implementiert. Eine Bewegung der Maus auf der x-Achse führt dabei zu einer Rotation der Kamera entlang der y-Achse und eine Bewegung auf der y-Achse zu einer Rotation entlang der x-Achse. Die Blickrichtung der Kamera folgt also effektiv der Bewegung der Maus. Dabei wurde das Drehen nur beim Gedrückthalten der Maustaste (dragging) durchgeführt, da die Kamera sich sonst natürlich bei der normalen Navigation auf der Seite bewegen würde. Das Gedrückthalten ist dabei schon weniger intuitiv, allerdings entspricht es der physischen Bewegung des Ziehens an einer Seite des Globus in der realen Welt. Hier ist allerdings eine genauere Evaluation der Steuerung notwendig um eine gute User-Experience zu gewährleisten.

Wichtig bei der Implementierung ist, dass die Geschwindigkeit der Bewegung nicht von der Geschwindigkeit des Browser abhängen darf. Daher müssen alle Vektoren, welche eine Bewegung darstellen, mit der Zeit multipliziert werden, die seit dem letzten Aufruf der Bewegung vergangen ist. Nur so wird in einem bestimmten Zeitabschnitt immer die gleiche Länge zurückgelegt. Ein weiterer Aspekt ist, dass das Gedrückthalten einer Taste natürlich zu einer kontinuierlichen Bewegung führen soll. Um dies zu erkennen kann zum einen das keydown-Event des Browser genutzt werden, welches auch gesendet wird, solange die Taste gedrückt bleibt. Allerdings ist dabei eine spürbare Verzögerung zwischen erstem und nachfolgenden Events vorhanden, sodass die Bewegung initial ziemlich ruckartig erfolgt. Auch ist die Geschwindigkeit, mit der die Events gesendet werden, nicht definiert und kann so zu sehr ruckartigen Bewegungen führen, sollte die Rate weit unter der Bildwiederholrate des Monitors liegen. Stattdessen werden die Kameras in der Render-Schleife, begrenzt durch die Bildwiederholungsrate (vSync), so lange in der gleichen Konfiguration geupdated, bis ein entsprechendes keyup-Event der gleichen Taste registriert wurde.

Die zweite Implementierung ist die stationäre Kamera, welche sich um den Globus in einem festen Abstand bewegt. Auch hier wurde die Rotation durch das Dragging der Maus implementiert. Da beide Kameras die selben Steuerungsmöglichkeiten besitzen sollte dies für den User schnell verständlich sein. Der Abstand zum Globus kann dabei mit dem Mausrad verändert werden, was auch sehr intuitiv sein sollte. Beim ersten Ansatz der Implementierung wurde die Kamera zum Rotierungspunkt (Pivot) bewegt, um den entsprechenden Winkel rotiert und anschließend im lokalen Koordinatensystem entlang des ursprünglichen Bewegungsvektors zurück bewegt. Ein wesentlich einfacher und genauerer Ansatz wurde dann in der Form des Scene-Graphs der THREE Bibliothek gefunden. Dabei wird eine Hierarchie an Objekten in Form einer Baumstruktur definiert und alle Transformationen eines Objektes werden automatisch an dessen Kinder weitergegeben. Wenn jetzt die Kamera als Kind des Pivot definiert wird, dann kann dieser normal rotiert werden und die Kamera rotiert automatisch im selben Winkel und Abstand mit. Für das Zoomen wurde das wheel-Event genutzt, welches Informationen darüber enthält, wie stark das Mausrad rotiert wurde. Hier gibt es die unterschiedlichen Einheiten Pixel, Zeilen und Seiten, welche auf die Navigation einer normaleren Seite mit Text ausgelegt sind. In konkreten Fall wurde nur die Scrollrichtung ermittelt und die Kamera mit einem festen Betrag entlang der lokalen z-Achse verschoben. Zusätzlich wurden minimale und maximale Abstände definiert und von der Implementation beachtet, damit die Kamera nicht über den Nullpunkt hinaus scrollt und somit effektiv die Scrollrichtung ändert.

Beide Implementationen updaten nach Erkennen jeglichen User-Inputs die View-Matrix und senden dann ein spezielles Event aus, dass den Ladeprozess\ref{dataLoading} in Gang setzt. Hier fiel auch ohne viel Testen auf, dass ein Anstoßen des Ladeprozesses bei jeder Interaktion performancetechnisch nicht durchführbar ist, da der Browser die Events in einer viel zu hohen Frequenz ausliefert. Es musste also eine Art Limitierung implementiert werden, die die Events auf eine maximale Rate beschränkt. Wichtig dabei war der Punkt, dass auch das letzte Event immer zugestellt werden musste. Es konnte also nicht einfach ein Zähler hochgezählt werden, der die vergangene Zeit inkrementiert und beim Erreichen des Limits das Events weiter propagiert, da natürlich jedes Event das potentiell letzte sein könnte. Stattdessen wurde die setTimeout-Funktion des Browsers genutzt, welche das Event am Ende der Wartezeit propagiert. Nachfolgende Events löschen dann den jeweils aktuellen Timeout und stellen sich selbst als Wartender in die virtuelle Warteschlange. Die Limitierung wurde im konkreten Fall auf 1 Event pro Sekunde gesetzt, was ein guter Kompromiss zwischen Performance und ausreichend dynamischem Ladens der Welt darstellt. Dieser feste Wert sollte idealerweise während der Laufzeit an die aktuelle Hardware angepasst werden um User Experience auf guter Hardware noch zu verbessern. Da dies jedoch nur eine fakultative Anwendung darstellt und das Finetuning zu viel Zeit in Anspruch genommen hätte, wurde es nicht implementiert.

\section{Daten-Ladeprozess}\label{dataLoading}

% beschreibe das Problem, dass Vertices benachbarter Abschnitte verbunden werden müssen, inklusive beider Bilder (x und y Achse)

\subsection{Modell Generierung}

\subsection{Projektion}

\section{User Interface}

% beschreibe hier vor allem, wie es alles zusammenhängt, also zum Beispiel die Inputs dann die Uniform Werte anpassen müssen oder zum Beispiel das Kamerascript ändern, was von der HRender-Schleife geupdated wird (wichtig dabei ist es, die jeweiligen Kamera-Scripte zu deaktivieren, da sie immer noch Nutzervenets erhalten)

\section{Tests}
Beim Testen der Anwendung hat dieses Projekt das gleiche Probleme aller Grafikprojekte: die Visualisierung kann ohne weiteres nicht automatisch getestet werden. Die Ausgabe auf dem Bildschirm ist zu komplex um daraus Testfälle zu definieren. Auch existieren zu viele Parameter, die das schlussendliche Ergebnis beeinflussen. Und natürlich könnte man Teile des Prozessen testen, zum Beispiel die Erstellung des 3D-Modells auf CPU Seite, allerdings ist es auch dort schwer erwartbare Testfälle zu definieren. Es ist einfach nicht definierbar, wie der Vertex auszusehen hat, der dem n-ten Höhenwert entspricht. Auch die Frage ob Formeln richtig implementiert wurden, lässt sich nur durch die Implementation der Formel im Test testen, was dem Sinn eines Tests widerspricht. Stattdessen wurden in diesem Projekt im Frontend viele manuelle Tests durchgeführt. Zum einen wurde die Visualisierung mit anderen Visualisierungen verglichen. Dabei reichte oft schon ein 2D Bild eines Referenzbildes einer verlässlichen Quelle, im konkreten Fall eine ähnlich eingefärbte Karte des gleichen Datensatzes vom JPL. Außerdem wurde mit Hilfe des Koordinaten und Höhenangaben Features bekannte geographische Features wie den Olympus Mons oder Hellas Planitia gegengeprüft. Da diese Features die Werte an Hand des Vertex berechnen und nicht die originalen Höhenwerte referenzieren, ist es eine recht gute Möglichkeit. Es ist sehr unwahrscheinlich, dass in beiden Implementationen Fehler enthalten sind, die sich gegenseitig aufheben zu richtigen Testresultaten führen. Ansonsten wurde zum Beispiel manuell sichergestellt, dass keine Lücken in der Visualisierung zu sehen sind oder dass die Panels wie geplant aus- und einklappen können. Auch die unterschiedlichen Kamerabewegungen wurden manuell überprüft. Nur bei der Konfiguration der maximalen und minimalen Höhe für die Interpolation wäre ein automatischer Test für das Verhalten bei falschen Benutzereingaben denkbar gewesen. 

Im Backend dagegen wurde jede öffentliche Methode zumindest mit einem Testfall abgedeckt. Relativ viel Aufwand wurde beim Testen der Redundanzentfernung betrieben. Hier wurde ein unbekannter Algorithmus getestet, der insbesondere keine optimale Lösung finden kann. Dadurch wurden viele Testfälle definiert, die prüfen, ob sehr einfache Optimierungsfälle, wie zum Beispiel die Überprüfung aller Alternativen bei nachfolgenden linearen Abhängigkeiten, eingehalten wurden. Auch Edge Cases konnten hier schön abgedeckt werden, zum Beispiel Fälle von Redundanzen die am Rand eines Abschnitts liegen oder bei denen redundante Teilraster nebeneinander liegen.
