\chapter{Kurzbeschreibung}
Mit dieser Arbeit werden Probleme bei der 3D Visualisierung von Geländemodellen in höchster Auflösung am Beispiel des Mars erforscht. Insbesondere steht dabei
der hohe Speicherverbrauch im Vordergrund, der für einen durchschnittlichen Arbeitsspeicher zu groß ist. Es werden daher verschiedene Methoden, sowohl verlustfrei als auch
verlustbehaftet, implementiert und gegeneinander evaluiert. Die Evaluation findet dabei an Hand eines Prototypen statt, der als interaktive Web-Anwendung realisiert wurde. Als Evaluationsmethoden kommt zum einen eine Nutzerbefragung zum Einsatz, zum anderen wird der Erfolg durch quantitative Methoden ermittelt. Alle Schritte der Entwicklung wurden dabei dokumentiert, sodass gefundene Lösungen dabei ähnlichen Projekten als Hilfe dienen können. Eine Kernentscheidung war dabei die starke Reduktion des Arbeits- und Grafikspeichers, unter anderen durch das Auslagern der Quelldaten außerhalb des Arbeitsspeichers. Dies führt zu einer Unabhängigkeit gegenüber des Datensatzes und erlaubt in der Theorie auch eine Nutzung von deutlich größeren Datenmengen. Dies führte durch die Nutzung des Webs allerdings auch zu einer starken Begrenzung der Performance durch das Netzwerk. Die Nutzung von verlustfreien Methoden, im konkreten Redundanzentfernungen und \textit{frustum-} und \textit{occlusion culling}, war erfolgreich und sollte in allen Visualisierungen von rasterbasierten Geländemodellen (DEM) zum Einsatz  kommen. Allerdings fiel durch die Nutzerevaluation auf, dass eine 3D Visualisierung aus Nutzersicht gar nicht notwendig war. Hier sollten ähnliche Projekte die Nutzung einer Textur auf einem geringer aufgelöstem 3D Modell als Alternative in Betracht ziehen. Auch die Wahl des Webs als Zielsystem stellte das Projekt vor einige Herausforderungen und sollte, trotz einiger Vorteile, gut bedacht werden. Die Datenmenge konnte, abschließend betrachtet, dank der hier beschriebenen Methoden jedoch bewältigt werden.
