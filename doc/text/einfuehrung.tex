\chapter{Einführung}

\section{Motivation}
Der Mars als unser Nachbarplanet ist, aufgrund seines Klimas und der Nähe, der am Besten für eine Besiedelung geeignetste, erdähnliche Planet in unserem Sonnensystem. Auch ist er das Ziel unzähliger Forschungen, insbesondere zu der Frage ob Leben außerhalb der Erde existieren kann. Aktuell findet ein kleiner Wettlauf zum Mars statt. Der \textit{Perseverance Rover} der NASA landete am 18.02.2021 und \textit{Tianwen-1} aus China am 14.05.2021. Des Weiteren ist derzeit die Raumsonde \textit{al-Amal} aus den Vereinigten Arabischen Emiraten in der Mars-Umlaufbahn, eine Landung ist allerdings nicht vorgesehen. Des Weiteren planen auch Japan und Indien eine Rover Mission in den nächsten Jahren. Allerdings sind Mars-Missionen kein einfaches Unterfangen, von den letzten 50 Mission schlugen 29 zumindest teilweise fehl\cite{marsMissions}. Eine Visualisierung hat vor allem informative Gründe. Dieses Projekt versucht interessierten Menschen zum Beispiel die Probleme einer Mars-Besiedelung oder allein schon die Probleme einer Landung vor Augen zu führen. Vor allem, da zwar einige 2D Visualisierungen mit derselben Datengrundlage existieren, echte 3D Visualisierungen aber schwer zu finden sind.

\section{Zielstellung}
Im Zuge dieser Arbeit soll ein Prototyp geschaffen werden, welcher die Oberflächenstruktur des Mars in höchstem Detailgrad als 3D Modell darstellt. Der Nutzer soll möglichst ungehindert die Oberfläche aus verschiedenen Winkeln und Zoomstufen betrachten können. Ein kleines User Interface soll Informationen über den aktuellen Ort darstellen und den Nutzer bei der Navigation unterstützen.

Die Frage, die diese Arbeit mit dem Projekt zu beantworten versucht, ist, welche technischen Möglichkeiten existieren um mit dem Problem der Datenmenge (siehe Abschnitt \ref{datenmenge}) fertig zu werden und ob eine so detailgetreue Darstellung aus Nutzersicht überhaupt sinnvoll ist. Vor allem, da eine Verbesserung des Detailgrades natürlich immer mit einer Verschlechterung der Performance in Verbindung gesetzt werden muss. Hierbei sollen verschiedene Möglichkeiten, sowohl verlustfrei als auch verlustbehaftet, implementiert und durch verschiedene Methoden evaluiert werden.

\section{Aufbau der Arbeit}
Im Laufe der Entwicklung dieser Arbeit wird ein Prototyp entwickelt. Zum besseren Verständnis kann es in einigen Abschnitten auch hilfreich sein, ihn mit der finalen Version zu vergleichen, welche unter \url{http://solarviewer.ddns.net:8080} zur Verfügung gestellt wird. Für die dauerhafte Verfügbarkeit und Aktualität kann nicht garantiert werden. Auch der Quelltext dieses Projekts ist öffentlich verfügbar und wird unter \url{https://github.com/Tasemo/SolarViewer} bereitgestellt. Die dort befindliche Anleitung kann auch dazu genutzt werden, die Anwendung lokal laufen zu lassen, sollte der erstgenannte Link nicht mehr funktionieren. Eine digitale Version dieser Arbeit ist des Weiteren unter \url{https://github.com/Tasemo/SolarViewer/blob/master/doc/thesis.pdf} zu finden.

In Kapitel \ref{chapter2} werden die notwendigen Grundlagen dargestellt. Als erstes werden die Datenquelle, das Datenformat und die Limitationen des Datensatzes erklärt. Anschließend werden kurz die für das Verständnis notwendigen Grundlagen von OpenGL und der programmierbaren Grafik-Pipeline vermittelt. Des Weiteren wird die zu erwartende Datenmenge berechnet um die Notwendigkeit von Reduktionsmethoden zu verdeutlichen. Diese werden im Anschluss vorgestellt, wobei verlustfreie als auch verlustbehaftete Methoden recherchiert werden. Auch einige Alternativen zur genutzten Methode werden kurz vorgestellt. Im Speziellen wird die Problematik der Redundanzentfernung vorgestellt, da dies im Kontext von Geländemodellen einen hohen Stellenwert einnimmt.

In Kapitel \ref{chapter3} werden die Anforderungen analysiert und dabei als erstes die Zielgruppe dieses Projekts definiert. Mit dieser Information werden bekannte Visualisierungen analysiert, die ebenfalls auf diese Zielgruppe ausgelegt sind. Dabei werden insbesondere deren Stärken und Schwächen hervorgehoben, da diese als Orientierung für dieses Projekt dienten. Mit diesen Informationen werden dann sowohl funktionale als auch nicht-funktionale Anforderungen definiert.

In Kapitel \ref{chapter4} werden alle Designentscheidungen dieses Projekts besprochen. Dabei wird als erstes das verwendete Vorgehensmodell beschrieben. Anschließend findet eine Evaluation des Zielsystems und der Programmiersprache statt. Dies hat, insbesondere auf Grund der Datenmenge, einen großen Einfluss auf den Erfolg dieses Projekts. Auch die verwendeten Frameworks werden hier analysiert. Des Weiteren wird die Architektur der Anwendung definiert, wobei insbesondere das Verständnis des Ladeprozess von großer Wichtigkeit ist. Abschließend wird das User Interface geplant.

In Kapitel \ref{chapter5} wird besprochen, auf welche Weise die Anforderungen umgesetzt wurden und welche Probleme bei der Implementierung auftraten. Dabei wird als erstes das Build System vorgestellt, da es bei diesem Projekt eine hohe Komplexität erreicht hat. Anschließend wird die Visualisierung auf Client-Seite beschrieben und welche Kamera Implementationen implementiert wurden. Anschließend werden die einzelnen Schritte des Ladeprozessen besprochen, wobei auch hier Limitationen auf Grund der Datenmenge beschrieben werden. Des Weiteren werden die zusätzlichen Funktionen auf Client-Seite und das Problem des Testens von Grafikanwendungen erläutert.

In Kapitel \ref{chapter6} wird die Implementierung dann evaluiert. Hier kommt zum einen eine Evaluation der User Experience an Hand einer Nutzerbefragung zum Einsatz. Zum anderen wird ein umfassendes Profiling durchgeführt, um sicherzustellen, dass die Anwendung mit der Datenmenge keine Probleme hat.

In Kapitel \ref{chapter7} wird dann die durchgeführte Arbeit zusammengefasst. Dabei werden die wichtigsten Erkenntnisse hervorgehoben. Auch Fehler und Limitationen der Anwendung werden dort beleuchtet. Abschließend werden weitere Entwicklungsperspektiven dargelegt.
