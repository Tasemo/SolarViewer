\chapter{Einführung}

\section{Motivation}
Der Mars als unser Nachbarplanet ist, aufgrund seines Klimas und der Nähe, der am Besten für eine Besiedelung geeignetste, erdähnliche Planet in unserem Sonnensystem. Auch ist er das Ziel unzähliger Forschungen, insbesondere zu der Frage ob Leben außerhalb der Erde existieren kann. Aktuell findet ein kleiner Wettlauf zum Mars statt. Der \textit{Perseverance Rover} der NASA landete am 18.02.2021 und \textit{Tianwen-1} aus China am 14.05.2021. Des Weiteren ist derzeit die Raumsonde \textit{al-Amal} aus den Vereinigten Arabischen Emiraten in der Mars-Umlaufbahn, eine Landung ist allerdings nicht vorgesehen. Des Weiteren planen auch Japan und Indien eine Rover Mission in den nächsten Jahren. Allerdings sind Mars-Missionen kein einfaches Unterfangen, von den letzten 50 Mission schlugen 29 zumindest teilweise fehl\footnote{\url{https://de.wikipedia.org/wiki/Chronologie_der_Marsmissionen}}.Eine Visualisierung hat vor allem informative Gründe. Dieses Projekt versucht interessierten Menschen zum Beispiel die Probleme einer Mars-Besiedelung oder allein schon die Probleme einer Landung vor Augen zu führen. Vor allem, da zwar einige 2D Visualisierungen mit derselben Datengrundlage existieren, echte 3D Visualisierungen aber schwer zu finden sind.

\section{Zielstellung}
Im Zuge dieser Arbeit soll ein Prototyp geschaffen werden, welcher die Oberflächenstruktur des Mars in höchstem Detailgrad als 3D Modell darstellt. Der Nutzer soll möglichst ungehindert die Oberfläche aus verschiedenen Winkeln und Zoomstufen betrachten können. Ein kleines User Interface soll Informationen über den aktuellen Ort darstellen und den Nutzer bei der Navigation unterstützen.

Die Frage, die diese Arbeit mit dem Projekt zu beantworten versucht, ist, welche technischen Möglichkeiten existieren um mit dem Problem der Datenmenge (siehe Abschnitt \ref{datenmenge}) fertig zu werden und ob eine so detailgetreue Darstellung aus Nutzersicht überhaupt sinnvoll ist. Vor allem, da eine Verbesserung des Detailgrades natürlich immer mit einer Verschlechterung der Performance in Verbindung gesetzt werden muss. Hierbei sollen verschiedene Möglichkeiten, sowohl verlustfrei als auch verlustbehaftet, implementiert und durch quantitative als auch qualitative/empirische Methoden evaluiert werden.

\section{Aufbau der Arbeit}
% Link zum Prototyp und seinem Quellcode
% bechreibe, dass keine Garnatiee für die ständige Verfügbarkeit gegegen werden kann, Anleitung zum selbst kompilieren dazu packen (Link zum Download der Daten)
% Java 15 JDK, (headless version ist ausreichend), Maven, Git, Download der Daten unter: https://planetarymaps.usgs.gov/mosaic/Mars_MGS_MOLA_DEM_mosaic_global_463m.tif, packen in einen Ordner "/data" im Rootverzeichnis des Projekts, mvn package aus dem Rootverzeichnis, Ausführen mit: java -Xmx4G -jar target/solar-viewer-1.0-SNAPSHOT-jar-with-dependencies.jar
% Link zur virtuellen Version dieses Dokuments
